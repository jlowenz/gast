\usepackage{graphicx}
\usepackage{savesym} \savesymbol{iint} \savesymbol{iiint} \usepackage{amsmath}  \usepackage{amssymb} 
\usepackage{color} 
\restoresymbol{T}{iint} \restoresymbol{T}{iiint} 
\usepackage{xifthen}
\usepackage{mathtools}
\usepackage{flushend}
\usepackage{marginnote}
\usepackage{setspace}
\usepackage[utf8]{inputenc}
\usepackage{tabularx}
\usepackage[hidelinks]{hyperref}
\usepackage{booktabs}
\usepackage{xfrac}
\usepackage{xparse}
\usepackage{threeparttable}
\usepackage[acronym,shortcuts]{glossaries}
\usepackage{needspace}
\usepackage{pifont}
\newcommand{\cmark}{\ding{51}}%
\newcommand{\xmark}{\ding{55}}%
%\usepackage{acro}
%\acsetup{first-style=short}
\usepackage{listings}
\usepackage{xcolor,colortbl}
\usepackage{rotating}
\usepackage{adjustbox}
\usepackage{algorithm}
\usepackage{algpseudocode}


%% Adjust captions to meet ARL standards
\usepackage{subcaption}
\captionsetup[figure]{labelfont=bf,textfont=bf,size=footnotesize,labelsep=space}
\captionsetup[subfigure]{labelfont=bf,textfont=bf,size=footnotesize,labelsep=space}
\captionsetup[table]{labelfont=bf,textfont=bf,size=footnotesize,labelsep=space}
\usepackage{ARLboxhandler}

%% Make tables smaller
\newcommand*{\smalltables}{\let\Tabular\tabular
\def\tabular{\sffamily\fontsize{7.000000}{8.400000}\selectfont\Tabular}
\definecolor{blue}{rgb}{0.316186,0.433203,0.916155}
\definecolor{lightblue}{rgb}{0.716186,0.833203,0.916155}}
\newcommand*{\restoretables}{\let\tabular\Tabular}

%%%%%%%%%%%%%%%%%%%%%%%%%%%%%%%%%%%%%%%%%%%%%%%%%%%%%%%%%%%%%%%%%%%%%%%%%%%%%%%%
%% support creating braces INSTEAD of parens when already inside parens
\newcommand*{\pac}[2][]{\ifglsused{#2}{\acs[#1]{#2}}{%
 \glsunset{#2}%
 \acl[#1]{#2} [\acs[#1]{#2}]}}

\newcommand{\ra}[1]{\renewcommand{\arraystretch}{#1}}
\ra{1.3}

%% COMMENT COMMANDS
\newcommand{\comment}[3]{\marginpar{\raggedright\setstretch{0.5}\textcolor{#1}{\scriptsize #2}}\textcolor{gray}{#3}} 
\newcommand{\hmmm}[2][]{\comment{orange}{#1}{#2}}
\newcommand{\jlo}[2][]{\comment{blue}{#1}{#2}}
\newcommand{\pro}[2][]{\comment{red}{#1}{#2}}
\newcommand{\jvp}[2][]{\comment{olive}{#1}{#2}}

%% FORMATTING

% ARL-compliant figure refs
\newcommand{\arlfig}[1]{Fig.~\ref{#1}}
\newcommand{\Arlfig}[1]{Figure~\ref{#1}}
\newcommand{\ie}{i.e.,~}
\newcommand{\eg}{e.g.,~}

%% MATH COMMANDS
\newcommand{\norm}[1]{\ensuremath{\left\lVert #1 \right\rVert}}
\renewcommand{\vec}[1]{\ensuremath{\mathbf{#1}}}
\newcommand{\neigh}[1]{\ensuremath{\mathcal{N}_{\vec{#1}}}}
\newcommand{\real}[1]{\ensuremath{\ifthenelse{\isempty{#1}}{\mathbb{R}}{\mathbb{R}^{#1}}}}
\newcommand{\Func}[1]{\ensuremath{\textbf{#1}}}
\renewcommand{\topfraction}{0.9}	% max fraction of floats at top
\renewcommand{\bottomfraction}{0.8}	% max fraction of floats at bottom
\newcommand{\vimg}{\ensuremath{\mathcal{I}}} 
\newcommand{\vrng}{\ensuremath{\mathcal{R}}} 
\newcommand{\vpcl}{\ensuremath{\mathcal{P}}}
\newcommand{\model}{\ensuremath{\mathcal{M}}}
\newcommand{\segments}{\ensuremath{\mathcal{S}}}
\newcommand{\vmask}{\ensuremath{\mathcal{V}}}
\newcommand{\true}{\ensuremath{\textbf{true}}}
\newcommand{\false}{\ensuremath{\textbf{false}}}
\newcommand{\normal}[1]{\ensuremath{\mathbf{\hat{#1}}}}
\newcommand{\normali}{\ensuremath{\normal{n}_i}}
\newcommand{\normalj}{\ensuremath{\normal{n}_j}}
\newcommand{\lseg}{\ensuremath{\vec{x}^j_i}}
\newcommand{\Rji}{\ensuremath{\prescript{i}{}{R_j}}}
\newcommand{\tji}{\ensuremath{\prescript{i}{}{\vec{t}_j}}}

\newcommand{\snode}{\ensuremath{\mathbb{S}}}
\newcommand{\lnode}{\ensuremath{\mathbb{L}}}
\newcommand{\pnode}{\ensuremath{\mathbb{P}}}
\newcommand{\sval}[1]{\ensuremath{\ifthenelse{\isempty{#1}}{\vec{x}^\snode}{\vec{x}_{#1}^\snode}}}
\newcommand{\lval}[1]{\ensuremath{\ifthenelse{\isempty{#1}}{\vec{x}^\lnode}{\vec{x}_{#1}^\lnode}}}
\newcommand{\pval}[1]{\ensuremath{\ifthenelse{\isempty{#1}}{\vec{x}^\pnode}{\vec{x}_{#1}^\pnode}}}
%\newcommand{\lval}[1]{\ensuremath{\vec{x}^\lnode}}
%\newcommand{\pval}[1]{\ensuremath{\vec{x}^\pnode}}
\newcommand{\typeof}[1]{\ensuremath{\mathrm{type}(#1)}}
\newcommand{\nodea}[1]{\ensuremath{\vec{x}_{#1}^\alpha}}
\newcommand{\nodeb}[1]{\ensuremath{\vec{x}_{#1}^\beta}}
\newcommand{\emean}[1]{\ensuremath{\vec{\mu}_{#1}}}
\newcommand{\proj}{\ensuremath{\pi}}
\newcommand{\pt}[1]{\ensuremath{\vec{p}_{#1}}}

\newcommand{\myparagraph}[1]{\textbf{\emph{#1}}.}

\renewcommand\[{\begin{equation}}
\renewcommand\]{\end{equation}}

%% environment for nice latex tables
%\newenvironment{asdtablewrap}[2]{%
%\newcommand{\hr}{\hline\addlinespace[0.1em]}
%\setlength{\tablecolsep}{12pt}}{}

% \thanks[*]{These authors have contributed equally to the content of this paper.}
%%%%%%%%%%%%%%%%%%%%%%%%%%%%%%%%%%%%%%%%%%%%%%%%%%
%% Commands for GAST
\newcommand{\Image}{\ensuremath{\mathcal{I}}}
\newcommand{\imgdom}{\ensuremath{\Omega}}
\newcommand{\integer}[1]{\ensuremath{\ifthenelse{\isempty{#1}}{\mathbb{Z}}{\mathbb{Z}^{#1}}}}
\newcommand{\var}[1]{{\ttfamily #1}}
\newcommand{\cmax}{C_{\max}}
\newcommand{\thetamax}{\theta_{\max}}
\newcommand{\pmin}{p_{\min}}
\newcommand{\pmax}{p_{\max}}
\newcommand{\T}{\ensuremath{^{\mkern1.5mu\mathsf{T}}}}
\newcommand{\atantwo}{\ensuremath{\operatorname{atan2}}}
\newcommand{\sym}{\ensuremath{M_\sigma(p)}}

\hyphenation{crystallography}

\lstdefinestyle{cuda}{
  language=C++,
  basicstyle=\footnotesize\ttfamily,
  keywordstyle=\bfseries\color{green!40!black},
  commentstyle=\itshape\color{purple!40!black},
  emph={__global__,blockIdx,blockDim,threadIdx,make_int2,int2,float2},
  emphstyle=\color{red!60!black},}
